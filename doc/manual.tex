\documentclass[11pt]{article}
%Gummi|065|=)
\title{\textbf{Welcome to Gummi 0.6.5}}
\author{Alexander van der Mey\\
		Wei-Ning Huang\\
		Dion Timmermann}
\date{}
\begin{document}

\maketitle

\section{Setup Guide}
\subsection{Clone Repository}

Prerequesities:
Git
Vagrant (Version)
Python 3 (Version)
miniconda
sudo apt-get install gdal-bin postgis

Switch to directory that should later contain your project-source-code

Clone Repository by typing into console:
%text{CODE: git clone https://<<your_username>>@bitbucket.org/kleingeist/spatial-weather.git}

Change directory to spatial-weather

CODE: cd spatial-weather

Install and configure the virtual machine by enter in console:

CODE: vagrant up

miniconda setup (aus mail)

--conda install --file requirements.conda
\\

muss jedes mal ausgeführt werden nachdem der server gestartet wurde.

activates Python virtual environment.

source ~/miniconda3/bin/activate spatial-weather

installs database-driver for python3:

sudo apt-get install python3-psycopg2 libpq-dev python3-dev

installs additional requirements for python:

-- requirements.txt

pip install -r requirements.txt


Create database and Postgis extesions:

%vagrant ssh -c "sudo -u postgres psql -c \"CREATE DATABASE spatial  OWNER myapp LC_COLLATE 'en_US.UTF-8' LC_CTYPE 'en_US.UTF-8';\"" 
%vagrant ssh -c "sudo -u postgres psql -d spatial -c \"CREATE EXTENSION postgis; CREATE EXTENSION postgis_topology;\"" 

create subfolder /data and copy gfs-data to that folder

%./run_gfs.py import data/ <startdate> <enddate>

Download http://download.geofabrik.de/europe/germany-latest.osm.pbf and save to data/ folder. (last access: 05.02.15)

Import OSM-data:





Start Webapp-Server:

python manage.py runserver

import (aus wiki)




\end{document}
