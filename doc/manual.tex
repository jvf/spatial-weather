\documentclass[paper=a4, fontsize=11pt]{scrartcl} % A4 paper and 11pt font size
\usepackage[utf8]{inputenc}
\usepackage[T1]{fontenc} % Use 8-bit encoding that has 256 glyphs
\usepackage{fourier} % Use the Adobe Utopia font for the document - comment this line to return to the LaTeX default
\usepackage[ngerman,british,UKenglish,USenglish,american]{babel}
\usepackage[utf8]{inputenc}
\usepackage{amsmath,amsfonts,amsthm} % Math packages
\usepackage{hyperref}
\usepackage{fancyhdr}
\usepackage{graphicx}
\usepackage{listings}
\usepackage{todonotes}
\usepackage{titlesec}
\usepackage{xcolor}
\definecolor{grey}{rgb}{0.4, 0.4, 0.4}

%\usepackage{sectsty} % Allows customizing section commands
%\allsectionsfont{\centering \normalfont\scshape} % Make all sections centered, the default font and small caps

\usepackage{fancyhdr} % Custom headers and footers
\pagestyle{fancyplain} % Makes all pages in the document conform to the custom headers and footers
\fancyhead[R]{} % No page header - if you want one, create it in the same way as the footers below
\fancyfoot[L]{} % Empty left footer
\fancyfoot[C]{} % Empty center footer
\fancyfoot[R]{\thepage} % Page numbering for right footer
\renewcommand{\headrulewidth}{0pt} % Remove header underlines
\renewcommand{\footrulewidth}{0pt} % Remove footer underlines
\setlength{\headheight}{13.6pt} % Customize the height of the header

%farbige Hyperlinks
%\definecolor{refcolor}{rgb}{0,.2,.4}
%schwarze Hyperlinks
\definecolor{refcolor}{rgb}{0,0,0}
%Hyperref Color
\hypersetup{pdftex=true, colorlinks=true, breaklinks=true, linkcolor=refcolor, menucolor=refcolor, pagecolor=refcolor, citecolor=refcolor, urlcolor=refcolor}

\numberwithin{equation}{section} % Number equations within sections (i.e. 1.1, 1.2, 2.1, 2.2 instead of 1, 2, 3, 4)
\numberwithin{figure}{section} % Number figures within sections (i.e. 1.1, 1.2, 2.1, 2.2 instead of 1, 2, 3, 4)
\numberwithin{table}{section} % Number tables within sections (i.e. 1.1, 1.2, 2.1, 2.2 instead of 1, 2, 3, 4)

\setlength\parindent{0pt} % Removes all indentation from paragraphs - comment this line for an assignment with lots of text

%----------------------------------------------------------------------------------------
%	TITLE SECTION
%----------------------------------------------------------------------------------------

\newcommand{\horrule}[1]{\rule{\linewidth}{#1}} % Create horizontal rule command with 1 argument of height

\title{	
\normalfont \normalsize 
\textsc{\includegraphics[width=0.6\textwidth]{pictures/logo} \\ [5pt] Arbeitsgruppe Datenbanken und Informationssysteme \\ [20pt] \includegraphics[width=0.15\textwidth]{pictures/DBIS_Logo_rgb_web.png}} \\ [10pt] % Your university, school and/or department name(s)
\horrule{0.5pt} \\[0.4cm] % Thin top horizontal rule
\huge Spatial Databases: Project Documentation \\ % The assignment title
\normalsize \textsc{Setup-Guide and Documentation for Spatial-Weather-Project} \\ [0.4cm]
\horrule{2pt} \\[0.5cm] % Thick bottom horizontal rule
}
\newcommand*{\justifyheading}{\raggedright}
\titleformat{\subsection}{\large\justifyheading}{\thesubsection}{1em}{}

\author{Johannes Dillmann (matr-nr) \\ Christian Wirth (4498611) \\ Jens Fischer (matr-nr)}

\date{\normalsize\today} % Today's date or a custom date
\begin{document}

\begin{titlepage}
\pagenumbering{Roman}
\maketitle
\thispagestyle{empty}
\end{titlepage}

\newpage
\tableofcontents
\thispagestyle{empty}

\newpage
\pagenumbering{arabic}
\pagestyle{fancy}
\setcounter{page}{1}

\section{Setup Guide}
\subsection{Prerequisites}

The Following Packages or Programs need to be installed before you proceed with this setup-guide:\\
Git
Vagrant
Python 3
PostGIS
miniconda

\begin{lstlisting}
sudo apt-get install gdal-bin postgis git 
\end{lstlisting}

\subsection{Step-by-step Setup}
Now switch to directory that should later contain your project-source-code

Clone Repository by typing into console:
\begin{lstlisting}
git clone https://<<your_username>>@bitbucket.org/kleingeist/spatial-weather.git
\end{lstlisting}

Change directory to spatial-weather
\begin{lstlisting}
cd spatial-weather
\end{lstlisting}

Install and configure the virtual machine by enter in console:
\begin{lstlisting}
vagrant up
\end{lstlisting}
miniconda setup (aus mail)
\todo[inline]{TDOD} 
--conda install --file requirements.conda

This command has to be executed every time the server has been started in order to activate the Python virtual environment.
\begin{lstlisting}
source ~/miniconda3/bin/activate spatial-weather
\end{lstlisting}

install database-driver for python3:
\begin{lstlisting}
sudo apt-get install python3-psycopg2 libpq-dev python3-dev
\end{lstlisting}

install additional requirements for python:\\
(take a look at requirements.txt for further details.)
\todo[inline]{TDOD}
\begin{lstlisting}
pip install -r requirements.txt
\end{lstlisting}

Create database and Postgis extesions:
\begin{lstlisting}
vagrant ssh -c "sudo -u postgres psql -c \"CREATE DATABASE spatial  OWNER myapp
LC_COLLATE 'en_US.UTF-8' LC_CTYPE 'en_US.UTF-8';\"" 
vagrant ssh -c "sudo -u postgres psql -d spatial -c \"CREATE EXTENSION postgis;
CREATE EXTENSION postgis_topology;\"" 
\end{lstlisting}
\subsection{Import Data}

%vagrant ssh -c "sudo -u postgres psql -c \"CREATE DATABASE spatial  OWNER myapp LC_COLLATE 'en_US.UTF-8' LC_CTYPE 'en_US.UTF-8';\"" 
%vagrant ssh -c "sudo -u postgres psql -d spatial -c \"CREATE EXTENSION postgis; CREATE EXTENSION postgis_topology;\"" 
%vagrant ssh -c "sudo -u postgres psql -d spatial -c \"GRANT ALL ON DATABASE spatial TO myapp; ALTER DATABASE spatial OWNER TO myapp; ALTER TABLE topology OWNER to myapp; ALTER TABLE layer OWNER to myapp; ALTER SCHEMA topology OWNER TO myapp;\""

create subfolder /data and copy gfs-data to that folder
\begin{lstlisting}
./run_gfs.py import data/ <startdate> <enddate>
\end{lstlisting}
Download http://download.geofabrik.de/europe/germany-latest.osm.pbf and save to data/ folder. (last access: 05.02.15)

Import OSM-data:

\todo[inline]{TODO}

Start Webapp-Server:
\begin{lstlisting}
python manage.py runserver
\end{lstlisting}

\todo[inline]{TODO: import (aus wiki)}
copy %germany_borders.geojson to data\ folder

run: %python manage.py import_dwd



\end{document}
